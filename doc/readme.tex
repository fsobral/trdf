% Readme 
% Conejo, Karas e Pedroso
% dicas para rodar nosso programa
% Abril 2015
% Free-derivative algorithm
\documentclass[12pt]{article} 

\usepackage[pdftitle={TRDF Tutorial},%
  breaklinks=true,%
  colorlinks=true,%
  linkcolor=blue,anchorcolor=blue,%
  citecolor=blue,filecolor=blue,%
  menucolor=blue,%
  urlcolor=blue]{hyperref}
  
  
\usepackage{geometry}
\usepackage{graphicx}
\usepackage{amssymb}
\usepackage{color}
\usepackage{amsmath}
\usepackage{indentfirst}
\usepackage{psfrag}
\usepackage{textcomp}
\usepackage{multirow}
\usepackage{amsmath}
\usepackage{longtable}
\usepackage{float}
\usepackage{caption} 
%\usepackage{wrapfig}
\usepackage[all]{xy}
\usepackage{hyperref}
% \usepackage{txfonts}
\bibliographystyle{plain}
%\usepackage{setspace}\onehalfspace
\newcommand{\HR}[2]{\href{#1}{#2}}
\newcommand{\pagina}{\HR{http://people.ufpr.br/~ewkaras/pesquisa/publicacoes/supplemental_conejo_karas_pedroso.html}{this
    page}}

%==================================================================
 
\begin{document}
\title{TRDF Tutorial\\A trust-region derivative-free algorithm for constrained optimization
}

\author{}

\maketitle


This text describes how to use the implemented algorithm TRDF proposed
in \cite{conejo2}. The algorithm solves the problem
\begin{equation}
\label{problem1}
\begin{array}{ll}
\text{minimize}  & \ f(x)\\
\mbox{subject to} & x\in\Omega
\end{array}\end{equation}
where $f:{\mathbb R}^n \to {\mathbb R}$ and $ \Omega=\{x\in{\mathbb
  R}^n \mid h(x)=0, g(x)\leq 0 \} $ where $h:{\mathbb R}^n\to{\mathbb
  R}^p$ and $g:{\mathbb R}^n\to{\mathbb R}^q$ are differentiable
functions. Although the algorithm can be applied when the objective
function is nonsmooth, it was designed for the class of problems in
which $f$ is smooth but its derivatives are not available.

The algorithm is implemented in {\tt Fortran} and is available at
\pagina \footnote{http://people.ufpr.br/$\sim$ewkaras/pesquisa/publicacoes/supplemental\_conejo\_karas\_pedroso.html}.

In Section~\ref{running}, we show how to compile and run a simple
problem for testing TRDF. Section~\ref{hs} is devoted to explaining
how to reproduce the numerical experiments with the Hock-Schittkowski
test problems. In sections~\ref{user} and~\ref{trdf} we explain how
the user can write his/her own problems and how to call the TRDF
subroutine using the correct parameters. Section~\ref{params} explains
the inner parameters defined in \texttt{tr\_params.par}. Finally, in
Section~\ref{solver} we describe how an advanced user can write an
interface and change the inner solver used by TRDF.

% To run a simple example of the algorithm in a Linux machine with
% \texttt{gfortran} and \texttt{gcc} packages, it is enough to download
% the folder TRDF and run
% \begin{verbatim}
% make trdf
% \end{verbatim}


% %These commands allow to run the algorithm with the  example test-problem.
% The  problem to be solved is defined in {\tt TRDF.f} and {\tt INIP.PROB} files:
% \begin{enumerate}
% \item In {\tt TRDF.f} file,  define:
% \begin{enumerate}
% \item  the size of the problem, {\tt N};
% \item the number of interpolation points {\tt NPT} (from $2\mbox{\tt N} + 1$ to $(\mbox{\tt N} + 1)(\mbox{\tt N} + 2) / 2$);
% \item the starting point {\tt X};
% \item  the box-constraints defined by lower and upper bounds, {\tt XL} and {\tt XU}. 
% \end {enumerate}
% \item In {\tt TRDF.f} file, edit the subroutine {\tt CALFUN} which defines the objective function.
% \item In {\tt INIP.PROB} file, define the constraints as in {\tt Algencan}. 
% %The subroutines that define the objective function, the gradient and the Hessian of the quadratic model are automatically updated.
%  Remember that the box-constraints were fixed in {\tt TRDF.f} file.
% \end{enumerate}

\section{Compiling and running}
\label{running}

The present implementation of the TRDF algorithm contains an example
problem (problem 37 from the Hock \& Schittkowski~\cite{hs} test
set). The file \texttt{trdf\_main.f} shows how to implement the
necessary subroutines, define the problem and call TRDF.

By default, the TRDF implementation uses the nonlinear programming
algorithm ALGENCAN~\cite{algencan,tango} for solving the constrained
trust-region subproblems. Two versions of ALGENCAN are supported:
2.2.1 and 3.0.0 (default). In order to use such versions, the user has
to download ALGENCAN and create the library
\texttt{libalgencan.a}. Version 3.0.0 already has a natural way to
create it, by simply typing \texttt{make} inside the top-level
directory of ALGENCAN. The library will be created at subdirectory
\texttt{lib}. See Section~\ref{solver} for more information on how to
write an interface to a different nonlinear programming solver.

After creating ALGENCAN's library, the user has to edit the
\texttt{Makefile} and change the path to the solver' library in
variable \texttt{SOLVERLIB}. Then, just type
\begin{verbatim}
make trdf
\end{verbatim}
to build the executable.

If version 2.2.1 is used, the user must manually build the library, by
entering in the (already downloaded) ALGENCAN directory and typing the
following commands:
\begin{verbatim}
gfortran -xf77-cpp-input -c *.f
ar ruv libalgencan.a *.o
\end{verbatim}
Then, the file \texttt{Makefile} should be edited to inform the path
to the library in variable \texttt{SOLVERLIB} \texttt{and} variable
\texttt{SOLVER} has to be changed to \texttt{algencan\_old\_solver}.

The user is able to suppress ALGENCAN's output by creating an empty
file called \texttt{.silent} in the same directory where the
executable will be run. This can be done by typing
\begin{verbatim}
touch .silent
\end{verbatim}
in the terminal. To suppress TRDF's output, please see
Section~\ref{params}.

Additionally, the present implementation also contains an interface to
the C language. Users familiar with the C language can write their
problems and solve using the TRDF algorithm. An example of how to
define a problem is implemented in file \texttt{trdf\_main.c}
(problem~37 of the Hock \& Schittkowski~\cite{hs} test set). The
process of building the libraries is the same for the \texttt{Fortran}
case, except that, the in the last step,
\begin{verbatim}
make c_trdf
\end{verbatim}
should be typed.

\subsection*{Quick start}
For quickly building the first example, under ALGENCAN 3.0.0, follow
the steps below:
\begin{enumerate}
\item Download and unpack ALGENCAN 3.0.0 in~\cite{tango};
\item In the root directory of ALGENCAN, build \texttt{libalgencan.a}
  by typing \texttt{make}. The file should be under \texttt{lib}
  subdirectory;
\item Download and unpack TRDF;
\item Under TRDF's directory, edit, in the file \texttt{Makefile}, the
  variable \texttt{SOLVERLIB} with the full path to the \texttt{lib}
  subdirectory of ALGENCAN;
\item Type \texttt{make trdf} ;
\item Suppress ALGENCAN's output by typing \texttt{touch .silent} ;
\item Run the TRDF example with \texttt{./trdf} .
\end{enumerate}

\section{Running the full Hock-Schittkowski test set}
\label{hs}

In order to reproduce the experiments of the supplemental material,
the user should build the \texttt{hstests} executable. For this task,
it is necessary first to build a library with the Hock-Schittkowski
test problems. The updated (Oct, 2011) source file of the problems can
be downloaded at
\begin{center}
\url{http://www.ai7.uni-bayreuth.de/tpnp08.htm}
\end{center}
but we provide an older version with TRDF (with some personal
corrections). The numerical experiments were performed with the
provided old version.

To build this executable is sufficient to build ALGENCAN's library, as
described in Section~\ref{running}, and then follow the steps below:
\begin{enumerate}
\item Type \texttt{make hstests} ;
\item Run the executable by typing \texttt{./hstests} ;
\item Write the problem number and hit Enter.
\end{enumerate}

An output file called \texttt{runhs.out} is created with the following
values: the problem number, number of variables, number of inequality
constraints, number of equality constraints, the best (known) value of
the objective function, the objective function value found by TRDF,
the sup-norm of the infeasibility (see Section~\ref{trdf}), and the
number of performed function evaluations.

We also provide a shell script \texttt{runhstests.sh} which runs
\underline{all} the test problems considered in the supplemental
material. In order to use this script, build the \texttt{hstests}
executable then type
\begin{verbatim}
./runhstests.sh hstests
\end{verbatim}
For each test problem, at most 30 minutes of CPU time will be
allowed. A file called \texttt{analysis-all} will be created at the
end of the tests, where each column was explained in the former
paragraph.


\section{The user-defined problem}
\label{user}

The user should define his/her own problem in a separate file. We
provide an possible example of implementation in file
\texttt{trdf\_main.f}. This file contains the main program where all
the input parameters are defined and the main TRDF subroutine is
called. Each TRDF parameter is fully detailed in
Section~\ref{trdf}. In addition, 4 user subroutines can be used by the
algorithm:
\begin{itemize}
\item \texttt{calobjf}: calculates the objective function
\item \texttt{calcon}: calculates the $i$-th constraint
\item \texttt{caljac}: calculates the sparse Jacobian of the $i$-th
  constraint
\item \texttt{calhc}: calculates the sparse Hessian of the $i$-th
  constraint
\end{itemize}

We suppose that at least two subroutines \textbf{have} to be coded by
the user: \texttt{calobjf} and \texttt{calcon} (if the problem has
constraints). If the problem does not have constraints, the user
should set $m = 0$ and create an empty subroutine.

Depending on the solver used for the subproblems (see
Section~\ref{solver}) the user can provide two more subroutines,
related with the first and second derivatives of the
constraints. ALGENCAN~\cite{tango}, the solver supported by default,
is able to use first and second order information. If subroutine
\texttt{caljac} is provided, the the user should set \texttt{CCODED(1)
  = .true.} in the main program. Likewise, if subroutine
\texttt{calhc} is provided, then \texttt{CCODED(2) = .true.} should be
set in the main program.

If the user does not want to use first and second order information to
solve his/her problem, both subroutines should still be present,
should be empty, return flag $-1$ and their respective \texttt{CCODED}
values should be set to \texttt{.false.}.

\section{The \texttt{TRDF} subroutine}
\label{trdf}

A full call to the TRDF subroutine is as follows:

\begin{verbatim}
      CALL TRDF(N,NPT,X,XL,XU,M,EQUATN,LINEAR,CCODED,
     +          MAXFCNT,RBEG,REND,XEPS,
     +          F,FEAS,FCNT)     
\end{verbatim}

Basically, the first line contains problem specific input parameters,
the second line contains configuration parameters (which can be
omitted by a call to the \texttt{EASYTRDF} subroutine) and the third
line contains the output parameters. Variable \texttt{X} is an
exception, since it is used as both input, for the starting point, and
output, for the solution found by the algorithm. The \texttt{EASYTRDF}
subroutine is an alias to the \texttt{TRDF} subroutine, where the
parameters \texttt{NPT}, \texttt{MAXFCNT}, \texttt{RBEG},
\texttt{REND} and \texttt{XEPS} are set to their default values.

Each one of the arguments is described below.

\begin{itemize}
\item \texttt{N} (integer, input): the dimension of the problem
\item \texttt{NPT} (integer, input): the number of points used for
  building the quadratic model of the objective function. Must be an
  integer in the range $[2N + 1,(N + 1)(N + 2) / 2]$
\item \texttt{X(N)} (double precision, input/output): as input,
  contains the initial point used for the subroutine. If the provided
  initial point is infeasible, the phase 0 is applied, an attempt to
  project it onto the feasible set $\Omega$ by calling \texttt{SOLVER}
  with \texttt{PHASE0 = .true.}. As output, it is the solution found
  by the algorithm
\item \texttt{XL(N)} and \texttt{XU(N)} (double precision, input): the
  lower and upper bounds on variable \texttt{X}, respectively
\item \texttt{M} (integer, input): the number of constraints of the problem
\item \texttt{EQUATN(M)} (logical, input): if \texttt{EQUATN(i)} is
  \texttt{.true.}, then constraint $i$ is an equality constraint
\item \texttt{LINEAR(M)} (logical, input): if \texttt{LINEAR(i)} is
  \texttt{.true.}, then constraint $i$ is a linear constraint
\item \texttt{CCODED(2)} (logical, input): indicates if the Jacobian
  (\texttt{CCODED(1) = .true.}) and the Hessian (\texttt{CCODED(2) =
    .true.}) of the constraints will be provided (see
  Section~\ref{user} for a description of the subroutines that should
  be provided by the user)
\item \texttt{MAXFCNT} (integer, input): the maximum number of allowed
  function evaluations
\item \texttt{RBEG} (double precision, input): the initial value of the
  trust-region radius
\item \texttt{REND} (double precision, input): the final value of the
  trust-region radius, used for declaring convergence of the method
\item \texttt{XEPS} (double precision, input): the tolerance used by
  the method to decide if a point is feasible or not
\item \texttt{F} (double precision, output): the value of the
  objective function at the solution
\item \texttt{FEAS} (double precision, output): the sup-norm of the
  infeasibility at the solution ($\max\{\|g(x^*)\|_\infty,
  \|h(x^*)\|_\infty\}$)
\item \texttt{FCNT} (integer, output): the number of function
  evaluations used to find the solution
\end{itemize}

\section{The file \texttt{tr\_params.par}}
\label{params}

The file \texttt{tr\_params.par} contains internal parameters used by
TRDF. They define the size of the arrays and also reduce the number of
parameters passed to the TRDF subroutine. Below we describe each
parameter in detail.

\begin{itemize}
\item \texttt{NMAX} is the maximum number of variables that are
  allowed. When solving problem with more than 1000 variables (or when
  working on a computer with very low CPU) it is necessary to change
  this value

\item \texttt{MMAX} is the maximum number of constraints. The same
  argument used for \texttt{NMAX} is applied here

\item \texttt{JCNNZMAX} is the maximum number of non null elements in
  the Jacobian of the constraints. The default value is set to be the
  full Jacobian (\texttt{NMAX} $\times$ \texttt{MMAX})

\item \texttt{HCNNZMAX} is the maximum number of non null elements in
  the Hessian of the objective function and of each constraint. The
  default value is set to be the case where all the elements of the
  Hessians are non null (\texttt{NMAX}$^2 \times (1 + $
  \texttt{MMAX}$)$)

\item \texttt{MAXXEL} is the maximum number of elements of the vector
  \texttt{X} that are displayed in the output

\item \texttt{INN} is a internal parameter that in most cases should
  have the same value of \texttt{NMAX}. So, if the user changes the
  value of \texttt{NMAX} it is recommended to also change this
  parameter

\item \texttt{OUTPUT} is a logical parameter that enables the output
  of TRDF if set to \texttt{.true.} and disables it, if set to
  \texttt{.false.}
\end{itemize}

\section{The \texttt{SOLVER} interface}
\label{solver}

At iteration $k$, $k = 1,\dots$, the algorithm needs to solve the
following constrained nonlinear subproblem:
\begin{equation}
  \label{problem2}
  \begin{array}{ll}
    \text{minimize}  & \ m_k(x)\\
    \mbox{subject to} & x\in\Omega \cap B_k
  \end{array}\end{equation}
where $m_k$ is the quadratic model for the objective function $f$ at
iteration $k$, $\Omega$ was defined in problem~\eqref{problem1} and
$B_k$ is the trust-region at iteration $k$.

The implementation of the TRDF algorithm is independent on the solver
used for the constrained trust-region subproblems. It is important to
note that problem~\eqref{problem2} has derivatives if the constraints
of the original problem~\eqref{problem1} have derivatives, since the
quadratic model has first and second derivatives.

In order to use a nonlinear solver for problem~\eqref{problem2}, it
is necessary to write an interface between TRDF and the desired
solver. This interface is given by subroutine \texttt{SOLVER}:

\begin{verbatim}
      SUBROUTINE SOLVER(N, L, U, X, M, EQUATN, LINEAR, CCODED,
     +                  PHASE0, EPS, CNORM, FLAG)
\end{verbatim}

The current implementation of TRDF provides two examples of
interfaces:\\ \texttt{algencan\_solver.f} is an interface for the
nonlinear solver ALGENCAN at version 3.0.0 and
\texttt{algencan\_old\_solver.f} is an interface for version 2.2.1. Each
time that TRDF subroutines call \texttt{SOLVER}, they expect that it
returns the solution \texttt{X}, its sup-norm of infeasibility
\texttt{CNORM} and a flag \texttt{FLAG}. If the solver finished
correctly, the flag must be 0. If problems have occurred during
optimization of~\eqref{problem2}, then the flag must be 2.

Below we describe each argument of the solver interface.

\begin{itemize}
\item \texttt{N} (integer, input): the dimension of the problem
\item \texttt{L(N)} and \texttt{U(N)} (double precision, input): the
  lower and upper bounds on variable \texttt{X}, respectively
\item \texttt{X(N)} (double precision, input/output): on entry, is the
  point used by the solver as starting point. On return, is the
  solution of problem~\eqref{problem2}
\item \texttt{EQUATN(M)}, \texttt{LINEAR(M)} and \texttt{CCODED(2)}
  (logical, input): this arguments have already been explained in
  Section~\ref{trdf}
\item \texttt{PHASE0} (logical, input): if \texttt{.true.} indicates
  that a projection onto $\Omega$ should be performed (thus, the
  objective function should be ignored), otherwise indicates that the
  full problem~\eqref{problem2} should be solved
\item \texttt{EPS} (double precision, input): the feasibility
  tolerance
\item \texttt{CNORM} (double precision, output): the sup-norm of the
  infeasibility at the solution found by the solver
\item \texttt{FLAG} (integer, output): indicates the status when
  returning. Must be $0$ if the solver converged correctly or $2$ if
  errors have occurred
\end{itemize}

After creating the specific solver interface, variables
\texttt{SOLVERLIB}, \texttt{SOLVER} and \texttt{SLOPTS} in
\texttt{Makefile} have to be changed, in order to use the new solver's
files and libraries when building TRDF.

\section*{More information}

If there are any doubts, fell free to send an email to Paulo
D. Conejo
\begin{center}
  \url{pconejo33@gmail.com}
\end{center}
or Francisco N. C. Sobral
\begin{center}
  \url{fncsobral@uem.br}
\end{center}


\begin{thebibliography}{10}
\bibitem{conejo2}
P.~D. Conejo, E.~W. Karas, L.~G. Pedroso.
\newblock  A trust-region derivative-free algorithm for constrained optimization.
\newblock {\em  Optimization Methods \& Software}, to appear, 2015.

\bibitem{algencan} R. Andreani, E. G. Birgin, J. M. Martínez, and
  M. L. Schuverdt.  \newblock On Augmented Lagrangian Methods with
  General Lower-Level Constraints \newblock SIAM Journal on
  Optimization, vol. 18, no. 4, pp. 1286–1309, 2008.

\bibitem{tango} ``Trustable Algorithms for Nonlinear General
  Optimization'',
  \HR{www.ime.usp.br/$\sim$egbirgin/tango}{www.ime.usp.br/$\sim$egbirgin/tango}.

\bibitem{hs} K. Schittkowski. \newblock Test Examples for Nonlinear
  Programming Codes - All Problems from the
  Hock-Schittkowski-Collection, 2009.
\end{thebibliography}

\end{document}
